\documentclass[12pt]{article}

\usepackage{fouriernc}
\usepackage{array} 

\begin{document}

\section*{Tips}
\begin{enumerate}

\item Distinguish between estimates, targets, and commitments.
\item When you're asked to provide an estimate, determine whether you're supposed to be estimating or figuring out how to hit a target.
\item When you see a single-point ``estimate,'' as if the number is an estimate or if it's really a target.
\item When you see a single-point estimate, that number's probability is not $100$\%. Ask what the probablity of that number is.
\item Don't provide ``percentage confident'' estimates (especially ``$90$\% confident'') unless you have a quantitatively derived basis for doing so.

\item Avoid using artificially narrow ranges. Be sure the ranges you use in your estimates don't misrepresent your confidence in your estimates.

\item If you are feeling pressure to make your ranges narrower, verify that the pressure actually is coming from an external source and not from yourself.

\item Don't intentionally underestimate. The penalty for underestimation is more severe than the penalty for overestimation. Address concerns about overestimation through planning and control, not by biasing your estimates.

\item Recongnize a mismatch between a project's business target and a project's estimate for what it is: valuable rist information that the project might not be successful. Take corrective actions early, when it can do some good. The possible corrective actions are:
\begin{itemize}
	\item Redefine scope of the project.
	\item Increase staff. Or, transfer best staff onto the project.
	\item Stagger the delivery of different functionality. Or decide that the project is not worth doing after all.
\end{itemize}

\item Many businesses value predictability more than development time, cost, or flexibility. Be sure you understand what your business values the most.

\item Consider the effect of the Cone of Uncertainty on the accuracy of your estimate. Your estimate cannot have more accuracy that is possible at your project's current position within the Cone. Stages in the Cone are:
	\begin{enumerate}
		\item Initial concept.
		\item Approved product definition.
		\item Requirements complete.
		\item User interface design complete.
		\item Detailed design complete.
		\item Software complete. 
	\end{enumerate}

\item Don't assume that the Cone of Uncertainty will narrow itself. You must force the Cone to narrow by removing sources of variability from your project.

\end{enumerate}

\section*{Takeaways}
\begin{itemize}
\item The primary purpose of software estimation is not to predict a project's outcome; it is to determine if a project's targets are realistic enough to allow the project to be controlled to meet them.

\item If you took a quiz with ten questions and you answered each question with $90$\% confidence, your chance of getting all ten correct is $34.9$\%. Your chance of getting nine of them correct is $38.7$\%. You have a chance of $93$\% for getting at least eight correct.

\item Most people's intuitive sense of ``$90$\% confident'' $\approx$ ``$30$\% confident.''

\item Developers typically estimate $20$\% to $30$\% lower than their actual effort.

\item The larger a project, in terms of LOC, the less chance the project has of completing on time and the greater chance it has of failing outright.

\item Software industry has an underestimation problem. Before we can make our estimates more accurate, we need to start making the estimates bigger.

\item Good estimates facilitate progress tracking through comparing planned progress against actual progress.

\item Projects that aim from the beginning to have the lowest number of defects usually also have the shortest schedules.

\item A project team that holds its ground and insists on an accurate estimate will improve its credibility within its organization.

\item There are typically 2-hour, 2-day, and 2-week versions of any particular feature.

\item The accuracy of the software estimate depends on the level of refinement of the software's definition. The more refined the definition, the more accurate the estimate.

\item Two approaches to account for the Cone:
	\begin{enumerate}
		\item Come up with a ``most likely'' estimate. Then use predefined multipliers to compute ranges.
				
				\begin{tabular}{lrrr}
			\firsthline
			\multicolumn{3}{r}{Scoping Error} \\
				\cline{2-4}
			Phase    & Low Side Error & High Side Error & Error Range \\
			\hline
			Initial Concept      & 0.25x (-75\%)    & 4x (+300\%) & 16x      \\
			Approved Definition       & 0.5x (-50\%)     & 2x (+100\%) & 4x      \\
			Requirements Complete       & 0.67\% (-33\%)     & 1.5x (+50\%)  & 2.25x     \\
			UI Design Complete & 0.8x (-20\%)      & 1.25x (+25\%) & 1.6x       \\
			Detailed Design Complete & 0.9x (-10\%) & 1.1x (+10\%) & 1.2x\\
\lasthline
		\end{tabular}
		
		\item One person estimates the best-case and worst-case ends of the range and a second person estimate the likelihood that the actual result will fall within that range.
	\end{enumerate}

\end{itemize}

\end{document}